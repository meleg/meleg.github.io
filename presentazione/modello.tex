\documentclass[intlimits]{beamer}

\mode<presentation>
{
  \usetheme{Madrid}
%  \usetheme{Frankfurt}
  \usecolortheme{seahorse}
  \usecolortheme{rose}
  \setbeamercovered{transparent}
  \setbeamertemplate{itemize item}[triangle]

  \setbeamertemplate{bibliography item}[book]

  \setbeamertemplate{navigation symbols}{}
  \setbeamertemplate{footline}
  {
    \leavevmode%
    \hbox{%
      \begin{beamercolorbox}[wd=.666666\paperwidth,ht=2.25ex,dp=1ex,center]{title in head/foot}%
        \usebeamerfont{title in head/foot}\insertshorttitle
      \end{beamercolorbox}%
      \begin{beamercolorbox}[wd=.333333\paperwidth,ht=2.25ex,dp=1ex,right] 
        {date in head/foot}%
        \usebeamerfont{date in head/foot}\insertshortdate{}\hspace*{2em}
        \insertframenumber{} / \inserttotalframenumber\hspace*{2ex}
    \end{beamercolorbox}}%
    \vskip0pt%
  }
}



% utf8 font & italian support
\usepackage[italian]{babel}
\usepackage{ucs}
\usepackage[utf8x]{inputenc}
\usepackage{lmodern}
\usepackage[T1]{fontenc}
\usepackage{textcomp}

%\usepackage{dsfont}
\usepackage{mathrsfs}

\numberwithin{equation}{section}
\theoremstyle{plain}
  \newtheorem{teor}{Teorema}[section]
  \newtheorem{prop}[teor]{Proposizione}
  \newtheorem{cor}[teor]{Corollario}
\theoremstyle{definition}
  \newtheorem{defin}[teor]{Definizione}
\theoremstyle{remark}
  \newtheorem{esempio}[teor]{Esempio}
  \newtheorem{esercizio}[teor]{Esercizio}
  \newtheorem{oss}[teor]{Osservazione}
  \newtheorem{prob}[teor]{Problema}

\newcommand{\gra}[1]{\left\{#1\right\}}
\newcommand{\pa}[1]{\left(#1\right)}
\newcommand{\ang}[1]{\left<#1\right>}
\newcommand{\qua}[1]{\left[#1\right]}
\newcommand{\abs}[1]{\left\lvert#1\right\rvert}
\newcommand{\norm}[1]{\left\lVert#1\right\rVert}
\newcommand{\normp}[1]{\norm{#1}_{p}}

%\renewcommand{\mathbb}{\mathds}


%\newcommand{\R}{\mathbb{R}}
%\newcommand{\N}{\mathbb{N}}

\newcommand{\Lp}{L^p(\Omega)}
\newcommand{\D}{\mathscr D}

\newcommand{\mb}[1]{\mathbf{#1}}
\newcommand{\x}{\mb x}
\newcommand{\ud}{\,\mathrm{d}}
\newcommand{\dx}{\ud\mb x}

\newcommand{\dotcup}{\ensuremath{\overset\circ\cup}}

\newcommand{\tc}{\ \textrm{:} \ }
\newcommand{\qo}{\quad \textrm{q.o.} \ }
\newcommand{\du}[1]{#1^\star}

\renewcommand{\leq}{\leqslant} 
\renewcommand{\geq}{\geqslant}
\renewcommand{\le}{\leqslant} 
\renewcommand{\ge}{\geqslant}
\renewcommand{\phi}{\varphi}
\renewcommand{\epsilon}{\varepsilon}

\DeclareMathOperator{\supp}{Supp}
\DeclareMathOperator{\diam}{diam}
\DeclareMathOperator{\diver}{div}

\title[Dim. Hausdorff e Ins. Furstenberg]
{Dimensione di Hausdorff\\ e Insiemi di Furstenberg}
\subtitle{Tesi di Laurea Triennale}

\author{Andrea Rossi}
\institute{Università di Pisa}
\date{30 settembre 2011}

% If you have a file called "university-logo-filename.xxx", where xxx
% is a graphic format that can be processed by latex or pdflatex,
% resp., then you can add a logo as follows:

\pgfdeclareimage[height=0.5cm]{university-logo}{../cherubino}
\logo{\pgfuseimage{university-logo}}


% Delete this, if you do not want the table of contents to pop up at
% the beginning of each subsection:
%\AtBeginSubsection[]
%{
%  \begin{frame}<beamer>{Outline}
%    \tableofcontents[currentsection,currentsubsection]
%  \end{frame}
%}


% If you wish to uncover everything in a step-wise fashion, uncomment
% the following command: 

%\beamerdefaultoverlayspecification{<+->}


\begin{document}

\begin{frame}
  \titlepage
\end{frame}



\section{Presentazione}
\begin{frame}{Indice.}
  \begin{itemize}[<+->]
    \item Presenteremo le definizioni classiche di:
      \begin{itemize}[<+->]
        \item Misure di Hausdorff.
        \item Dimensione di Hausdorff.
        \item Insieme di Furstenberg.
        \end{itemize}
    \item
      Estenderemo in modo opportuno il concetto di misure di Hausdorff.
    \item
      Mostreremo un insieme \textit{a-dimensionale} rispetto a queste misure di Hausdorff estese.
    \item
      Definiremo gli insiemi di Furstenberg generalizzati.
    \item
      Enunceremo due teoremi sugli insiemi di Furstenberg generalizzati.
    \item
      Come corollario otterremo un risultato importante per quanto riguarda
      la dimensione di un insieme di Furstenberg classico.
  \end{itemize}
\end{frame}





\section{Introduzione}
\begin{frame}{Misure esterne di Hausdorff $H_p^*$.}
\begin{defin}
Se $E \subseteq \mathbb{R}^N$, il \emph{diametro} di $E$ è il numero 
non negativo (eventualmente infinito)
  \[ diam\ E = \begin{cases}
    0 & \text{se $E = \emptyset$}\\
    \sup \gra{\abs{x- y} : x, y \in E} & \text{altrimenti}
.\end{cases}\]
\end{defin}
\pause
\begin{defin}
Sia $E \subseteq \mathbb{R}^N$, e siano $p,\,\delta \ge 0$. Definiamo
  \[ H^{*}_{p,\delta}(E) = \inf\gra{ \sum_{n \in \mathbb{N}}\pa{diam\ U_n}^p : U_n\ \text{aperti},\, diam\ {U_n} < \delta,\, E \subseteq \bigcup_{n \in \mathbb{N}}U_n}. \]
\pause
Sia $p > 0$, la \emph{misura esterna p-dimensionale di Hausdorff} $H^{*}_p$ è data da
\[ H^{*}_p(E) = \lim_{\delta\to0^+}H^{*}_{p,\delta}(E) = \sup_{\delta > 0}H^{*}_{p,\delta}(E)\   
\qquad   \forall E\subseteq \mathbb{R}^N.\]  
\end{defin}
\end{frame}


\begin{frame}{Proprietà delle misure esterne $H_p^*$.}
\begin{prop} Sia $p > 0$, allora: \pause
\begin{enumerate} 
\item  $H^{*}_p(E) \geq 0     \qquad \forall E\subseteq \mathbb{R}^N$; \pause
\item  $H^{*}_p(\emptyset) = H^{*}_p(\gra{x}) = 0  \qquad        \forall x\in \mathbb{R}^N$; \pause
\item $H^{*}_p$ è monotona e numerabilmente subadditiva;\pause
\item $H^{*}_p(E+x)=H^{*}_p(E)\qquad\forall x\in\mathbb{R}^N,\ \forall E \subseteq\mathbb{R}^N$;\pause
\item  $H^{*}_p(tE)= t^p H^{*}_p(E)\qquad\forall t>0,\ \forall E \subseteq\mathbb{R}^N$.
\end{enumerate}
\end{prop}  
\end{frame}


\begin{frame}{Misure di Hausdorff $H_p$.}
\begin{defin} La classe degli insiemi $H_p$-misurabili è 
\[M_H = \gra{E\subseteq \mathbb{R}^N\ :\ H^{*}_p(A) = H^{*}_p(A\cap E) + H^{*}_p(A\cap E^c)   
\qquad  \forall A\subseteq \mathbb{R}^N}.\] 
\end{defin}
\pause
\begin{defin} La misura di Hausdorff di indice $p$ è   $H_p =  H^{*}_p \vert_{M_H}$ .
\end{defin}
\pause
Si verifica che $M_H$ è una tribù e che $H_p$ è numerabilmente additiva sugli 
elementi disgiunti di $M_H$, quindi $H_p$ è effettivamente una misura.
\begin{prop} Per ogni $p > 0$,  si ha $B(\mathbb{R}^N) \subset M_H$,
ovvero i boreliani sono $H_p$-misurabili. \end{prop}
\end{frame}


\begin{frame}{Dimensione di Hausdorff}
Fissato $E \subseteq \mathbb{R}^N$, studiamo la misura esterna $H^{*}_p(E)$ al variare di $p > 0$.
\pause
\begin{prop} Per ogni $E\subseteq \mathbb{R}^N$ e per ogni $\epsilon$ positivo 
\[H^{*}_{N+\epsilon}(E)=0.\] \end{prop} 
Consideriamo quindi $H^{*}_p(E)$ limitandoci ai valori $p \in (0,N]$.
\pause
\begin{prop} Sia $E\subseteq \mathbb{R}^N$ e sia $p \in ]0,N]$: \\
\[\text{(i) se } H^{*}_p(E) < \infty  \quad \text{allora} \quad H^{*}_q(E) = 0   \qquad  \forall\ q \in\ ]p,N]; \]  
\[\text{(ii) se } H^{*}_p(E) > 0  \quad \text{allora} \quad H^{*}_q(E) = \infty  \qquad  \forall\ q \in\ ]0,p[.\]\end{prop}
\end{frame}
\begin{frame}{Dimensione di Hausdorff}
\begin{cor} Sia $E\subseteq \mathbb{R}^N$ tale che $H^{*}_p(E)$ non sia identicamente nulla per ogni $p > 0$, allora 
esiste $p_0 \in ]0,N]$ tale che \\
\[ H^{*}_p(E) = \begin{cases}
    \infty & \text{se $0 < p < p_0$ }\\
    \in [0,\infty] & \text{se $p = p_0$ }\\
    0 & \text{se $ p > p_0  $}
\end{cases}\]
\end{cor}
In particolare per ogni $E\subseteq \mathbb{R}^N$ la funzione $p\longmapsto H^{*}_p(E)$ risulta decrescente in $p$.
Di conseguenza possiamo introdurre il prossimo concetto. \\
\pause
\begin{defin} Si dice dimensione di Hausdorff di un sottoinsieme $E$ di $\mathbb{R}^N$, il numero
\[dim_H(E) = \inf\gra{p > 0:\ H^{*}_p(E) = 0}\ \in\ [0,N].\]
\end{defin}
\end{frame}
\begin{frame}{Dimensione di Hausdorff}
\begin{oss} Notiamo che il concetto di dimensione di Hausdorff vale per ogni $E\subseteq \mathbb{R}^N$, 
anche non misurabile, perché la definizione si basa sul concetto di misura esterna di Hausdorff, 
definita per ogni $E$. \\
Come da intuito, la dimensione di Hausdorff è monotona, ovvero $E\subset F$ implica $dim_H(E)\leq dim_H(F):$
questo deriva dal fatto che $H^{*}_p$ è monotona.\\
\pause
In pratica per dimostrare che un insieme dato $E$ ha una certa dimensione di Hausdorff $s$ è sufficiente
verificare che 
\[H^{*}_r(E) = \infty \qquad \forall r < s  \qquad \text{e} \qquad H^{*}_t(E) = 0 \qquad \forall t > s,\]
indipendentemente dal fatto che $H^{*}_s(E)$ sia nulla, finita positiva o infinita.
In particolare se $0<H^{*}_s(E)<\infty$ allora $dim_H(E)=s.$\\
\end{oss}
\end{frame}


\begin{frame}{Il rapporto tra la misura di Hausdorff e quella di Lebesgue}
Dopo aver introdotto la dimensione di Hausdorff, mostreremo ora come essa caratterizza gli insiemi
non trascurabili rispetto a $m_N$, ovvero quelli con misura di Lebesgue strettamente positiva.
\pause
\begin{teor} Esiste una costante $\alpha_N$, che dipende solo da $N$, 
tale che per ogni insieme $E\subseteq \mathbb{R}^N$ si ha
\[ H^{*}_N(E) = \alpha_N m^{*}_N(E).\]
\end{teor}
\pause
\begin{cor} Sia $E\subseteq \mathbb{R}^N$ tale che $m^{*}_N(E) > 0$, allora
\[dim_H(E) = N .\] 
\end{cor}
\end{frame}


\begin{frame}{Esempi}
In seguito al corollario limiteremo la nostra attenzione alla dimensione di Hausdorff degli insiemi di misura nulla 
secondo Lebesgue, e vedremo con alcuni esempi come tale dimensione riesca a "catalogare" e distinguere questi insiemi:
\pause
\begin{itemize}[<+->]
  \item sia $M \subset \mathbb{R}^N$ tale che $m^*_N(M)>0$, allora $M$ ha dimensione $N$;
  \item ogni insieme numerabile $E \subset \mathbb{R}^N$ ha dimensione 0;
  \item il supporto $\Gamma=\varphi([a,b])\subset \mathbb{R}^N$ di ogni curva semplice di classe $C^1$ ha dimensione 1;
\end{itemize}
\end{frame}


\begin{frame}{Un esempio di dimensione non intera: l'insieme di Cantor}
Sia $C\subset \mathbb{R}$ l'insieme di Cantor ottenuto togliendo da $[0,1]$ al primo passo
l'intervallo (aperto) centrale di ampiezza $1/3$, e ricorsivamente al $k$-esimo passo togliendo da ogni 
sottointervallo residuo l'intervallo (aperto) centrale di ampiezza $(1/3)^k$.
\begin{figure}
  \includegraphics[width=3in]{cantor}
  \caption{I primi cinque passi dell'iterazione.}  
\end{figure}
\pause
\begin{prop} L'insieme di Cantor ha dimensione $\frac{\log 2}{\log 3}$.\end{prop}
\end{frame} 


\begin{frame}{Insiemi di Furstenberg}
Presentiamo adesso una classe di particolari sottoinsiemi del
piano euclideo, la cui definizione si basa sul concetto di dimensione
di Hausdorff.
\pause
\bigskip
\begin{defin} Sia $\alpha \in ]0,1]$,
un sottoinsieme $E\subseteq \mathbb{R}^2$ si dice insieme di Furstenberg di 
tipo $\alpha$ (oppure insieme $F_\alpha$) se, per ogni direzione $e$ 
nel cerchio unitario, esiste un segmento $l_e$ nella direzione di $e$ tale che
$dim_H(l_e \cap E) \geq \alpha$. Diremo anche che in tal caso $E$ appartiene alla classe $F_\alpha$.\end{defin}
\end{frame}





\section{Estensione alle misure $H^h$}
\begin{frame}{Funzioni dimensione}
\begin{defin} Una funzione $h : [0,\infty) \longrightarrow [0,\infty)$ è chiamata funzione dimensione se
\[h(0) = 0,\ h(t)>0 \text{ per } t > 0,\ h \text{ è crescente e continua a destra.}\]
Denotiamo con $\mathbb{H}$ la classe delle funzioni dimensione.
\end{defin}
\pause
\begin{defin} Siano $g,h$ due funzioni dimensione. Diremo che $g$ è dimensionalmente più piccola
di $h$ e scriveremo $g \prec h$ se e solo se 
\[\lim_{x\to 0^+} \dfrac{h(x)}{g(x)} = 0 \]
\end{defin}
\end{frame}


\begin{frame}{Estensione}
Possiamo ora definire le misure esterne di Hausdorff $H^h$, dove $h \in \mathbb{H}$, analogamente al caso classico.
\bigskip
\begin{defin} Siano $E \subseteq \mathbb{R}^N$, $\delta > 0$ e $h$ una funzione dimensione. Definiamo
  \[ H^h_\delta(E) = \inf\gra{ \sum_{n \in \mathbb{N}}h(diam\ U_n): U_n\ 
  \text{aperti},\, diam\ {U_n} < \delta,\, E \subseteq \bigcup_{n \in \mathbb{N}}U_n} .\]
\pause
La misura esterna $h$-dimensionale di Hausdorff di $E \subseteq \mathbb{R}^N$ è
\[ H^h(E) = \lim_{\delta\to0^+}H^{h}_{\delta}(E) = \sup_{\delta > 0}H^{h}_{\delta}(E).\]\end{defin}
\end{frame}


\begin{frame}{Estensione}
\begin{oss} Notiamo come si tratti effettivamente di un'estensione delle definizioni precedenti,
nel senso che la definizione usuale di misura esterna di Hausdorff 
$H^{*}_p ( p > 0 )$ si ottiene per $h_p(x) = x^p$, che è proprio una funzione dimensione.
Analogamente per ogni $E\subseteq \mathbb{R}^N$ la funzione $h\longmapsto H^h(E)$ risulta "decrescente" in $h$,
ovvero se $g \prec h$ allora $H^h(E)\leq H^g(E)$.\end{oss}
\pause
Come vedremo, non è vero in generale che, dato un insieme $E$, esista una funzione $h \in \mathbb{H}$ tale che\\
\[ H^{g}(E) = \begin{cases}
    \infty & \text{se $g \prec h$ }\\
    0 & \text{se $ g\succ h$}
\end{cases}\]\\
\pause
ovvero con questa estensione alle misure $H^h$ perdiamo l'analogo della dimensione di Hausdorff,
che invece esiste per ogni insieme $E$ nella definizione standard.
\end{frame}


\begin{frame}{I risultati di Besicovitch}
\begin{teor} Sia $h \in \mathbb{H}$ tale che $H^{h}(E) = 0$, allora
\[ \exists\ g \prec h \ (g \in \mathbb{H}): \quad H^{g}(E) = 0.\]
\end{teor}
\bigskip
\pause
\begin{teor} Sia $E \subset \mathbb{R}^N$ un insieme boreliano. Se $h \in \mathbb{H}$ è tale 
che $E$ ha misura $H^h$ non $\sigma$-finita allora esiste
$g \succ h \ (g \in \mathbb{H})$ tale che $E$ ha misura $H^g$ non $\sigma$-finita.
\end{teor}
\end{frame}


\begin{frame}{I risultati di Besicovitch}
\begin{cor} Se per un boreliano $E \subset \mathbb{R^N}$ esiste $h \in \mathbb{H}$ tale che
\[H^g(E) > 0 \quad \forall g \prec h, \qquad \qquad H^g(E) = 0 \quad \forall g \succ h,\]
allora $E$ ha misura $H^h$ positiva e $\sigma$-finita.\end{cor}
\pause
\begin{defin} Se $E$ è un boreliano che verifica le ipotesi del corollario, esso si dice $h$-set; 
viceversa, se per ogni $h \in \mathbb{H} \ \ E$ non è un $h$-set,  $E$ si dice \textit{a-dimensionale}. \end{defin} 
\pause Il corollario è importante in quanto ci permetterà di mostrare
un esempio di insieme a-dimensionale: l'insieme dei numeri di Liouville.
\end{frame}


\begin{frame}{I numeri di Liouville}
\begin{defin} Chiamiamo insieme dei numeri di Liouville
\[ \mathbb{L} = \gra{x \in \mathbb{R}: \quad\forall n \in \mathbb{N}_0,\ \exists\ p,q \in \mathbb{Z} \ (q \geq 2):
\quad 0 < \abs{x - \frac{p}{q}} < \frac{1}{q^n}}.\]\end{defin}
\pause
\begin{prop} L'insieme $\mathbb{L}$ ha le seguenti proprietà:
\pause
\begin{enumerate}
\item $\emptyset \neq \mathbb{L} \subset \mathbb{Q}^c$, cioè i numeri di Liouville sono irrazionali; \pause
\item  $\mathbb{L}$ è un insieme $G_\delta$, quindi boreliano;\pause
\item $\mathbb{L}$ è denso in $\mathbb{R}$;\pause
\item  $\mathbb{L}$ ha misura $m$ di Lebesgue nulla; \pause
\item   $\mathbb{L}$ è periodico $mod\ \mathbb{Q}: \quad  \mathbb{L} + q = \mathbb{L}  \quad \forall\ q \in \mathbb{Q}$.  
\end{enumerate}
\end{prop}
\end{frame}


\begin{frame}{Misure di Borel su $\mathbb{R}$}
Per mostrare che $\mathbb{L}$ è un insieme a-dimensionale, dobbiamo utilizzare delle
interessanti proprietà di cui godono le misure di Borel su $\mathbb{R}$.\\
\bigskip
\pause
Indichiamo con $int(A)$ la parte interna dell'insieme $A$, 
definiamo $A+B =\gra{ a+b: a \in A, \quad b \in B}, \quad A+t=\gra{ a+t:a \in A}$.\pause
\begin{teor}Sia $B\subset \mathbb{R}$ un boreliano di misura di Lebesgue nulla e $\mu$ una misura boreliana 
(cioè definita su una tribù contenente i boreliani) su $\mathbb{R}$ tale che $B$ abbia misura $\mu$ positiva e $\sigma$-finita. 
\pause Allora
esiste un compatto $C \subset B \quad$ con $\mu(C)>0$ e $int(C-C)=\emptyset$.\end{teor}\pause
\begin{teor}Sia $B$ un insieme $G_\delta$ denso tale che $\gra{t \in \mathbb{R}: B+t \subset B}$ sia denso in $\mathbb{R}$
e sia $C \subset B$ un compatto con $int(C-C)=\emptyset$. \pause Allora  $B$ contiene una quantità più 
che numerabile di traslati disgiunti di $C$.\end{teor}
\end{frame}


\begin{frame}{Misure di Borel su $\mathbb{R}$}
Grazie ai due teoremi appena enunciati otteniamo
\begin{teor}Sia $B\subset \mathbb{R}$ un insieme (non vuoto) $G_\delta$ di misura di Lebesgue nulla,
e supponiamo che $\gra{t \in \mathbb{R}: B+t \subset B}$ sia denso in $\mathbb{R}$. \pause Allora per ogni misura 
$\mu$ boreliana su $\mathbb{R}$ e invariante per traslazioni si ha \pause o $\mu(B)=0$ 
oppure $B$ ha misura $\mu$ non $\sigma$-finita.\end{teor}
\bigskip
\pause
\begin{cor}$\mathbb{L}$ è un insieme a-dimensionale.\end{cor}
\end{frame}





\section{Insiemi di Furstenberg generalizzati}
\begin{frame}{Insiemi di Furstenberg generalizzati}
\begin{defin} Sia $h \in \mathbb{H}$, un sottoinsieme $E\subseteq \mathbb{R}^2$ si dice \textit{insieme di Furstenberg di 
tipo $h$}, o insieme $F_h$, se per ogni direzione $e \in \mathbb{S}^1$ esiste un segmento $l_e$ nella 
direzione di $e$ tale che $H^h(l_e \cap E) >0$.\end{defin}
\end{frame}


\begin{frame}{Insiemi di Furstenberg generalizzati}
\begin{defin} \[\mathbb{H}_d = \gra{h \in \mathbb{H}: \ h(2x)\leq Ch(x) \ \text{ per qualche } C>0}.\] \end{defin}\pause
\begin{defin} Date due funzioni $g,h \in \mathbb{H}$ definiamo
\[ \Delta_0(g,h)(x)=\frac{g(x)}{h(x)}, \qquad \Delta_1(g,h)(x)=\frac{g(x)}{h^2(x)}.\] \end{defin}
\end{frame}


\begin{frame}{Condizioni sufficienti affinché $H^g(F_h)>0$}
\begin{teor} Siano $E$ un insieme $F_h$ con $h \in \mathbb{H}_d$, $g \in \mathbb{H}$ tale che $g \prec h^2$. \pause Posto
\[a_k= \sqrt{\frac{k}{\Delta_1(g,h)(2^{-k})}},\] supponiamo che la serie $\sum_{k=0}^\infty a_k$ converga. \pause
 Allora \[H^g(E)>0.\]
\end{teor}
\end{frame}


\begin{frame}{Condizioni sufficienti affinché $H^g(F_h)>0$}
\begin{teor} Sia $E$ un insieme $F_h$, con $h \in \mathbb{H}_d$, tale che $h(x) \lesssim x^\alpha$ per qualche 
$0< \alpha <1$, e sia $g \in \mathbb{H}$ tale che $g \prec h$. \pause 
  Posto \[a_k= (\Delta_0(h,g)(2^{-k}))^{\frac{2\alpha}{2\alpha+1}},\] 
se la serie $\sum_{k=0}^\infty a_k$ converge, \pause
allora \[H^{g \sqrt{\cdot}}(E)>0. \]
\end{teor}
\end{frame}


\begin{frame}{Il caso degli insiemi di Furstenberg classici}
Riprendiamo la definizione di insieme di Furstenberg classico.\\
\begin{defin}Sia $\alpha \in ]0,1]$, un sottoinsieme $E\subseteq \mathbb{R}^2$ si dice insieme di Furstenberg di 
tipo $\alpha$ (oppure insieme $F_\alpha$) se, per ogni direzione $e$ 
nel cerchio unitario, esiste un segmento $l_e$ nella direzione di $e$ tale che
$dim_H(l_e \cap E) \geq \alpha$. Diremo anche che in tal caso $E$ appartiene alla classe $F_\alpha$.\end{defin}
\pause \begin{oss}
Un insieme di Furstenberg classico di tipo $F_\alpha$ con $\alpha \in \ ]0,1]$ non è altro che un 
caso particolare di insieme di Furstenberg generalizzato
di tipo $F_h$, con $h(x)= x^\alpha$. \end{oss}
\end{frame}


\begin{frame}{Il teorema di Wolff}
Useremo i due importanti teoremi sugli insiemi di Furstenberg generalizzati per dare un risultato notevole,
dovuto a Wolff, per quanto riguarda la dimensione di Hausdorff di un insieme del tipo $F_\alpha$.\pause
\bigskip
\begin{teor}[Wolff] Dato $\alpha \in ]0,1]$, sia $E \in F_\alpha$. \pause Allora
\[dim_H(E) \geq \max\gra{2\alpha,\alpha + 1/2}.\] \end{teor}
\end{frame}


\begin{frame}{Un'applicazione del teorema di Wolff}
Applichiamo infine il teorema di Wolff per stimare la
dimensione di un opportuno insieme di Furstenberg che andiamo a costruire:\pause
\begin{esempio} Consideriamo il classico insieme di Cantor $C\subset \mathbb{R}$,
e sia $C^{'}= C- 1/2 \subset [-1/2, 1/2]$ il suo traslato verso sinistra di $1/2$. \pause
 Definiamo  \[\mathcal{W}=\gra{(x,y):\ x=r\cos\theta,\ y=r\sin\theta,\ r \in C^{'},\ \theta \in [0,\pi]}\subset \mathbb{R}^2.\] \pause
$\mathcal{W}$ contiene in ogni direzione un insieme di Cantor, \pause
quindi $\mathcal{W}$ è un insieme di Furstenberg $F_\alpha$ dove $\alpha=\frac{\log 2}{\log 3}$. \pause
Allora per il teorema di Wolff si ha
\[dim_H(\mathcal{W}) \geq \max\gra{2\alpha,\alpha+ \frac{1}{2}} = \max\gra{\frac{2\log 2}{\log 3},
\frac{\log 2}{\log 3}+ \frac{1}{2}} = \frac{\log 4}{\log 3}.\]
\end{esempio}
\end{frame}





\begin{frame}{Conclusioni}
\end{frame}





% All of the following is optional and typically not needed. 
\appendix
\section<presentation>*{\appendixname}
\subsection<presentation>*{Bibliografia essenziale}
%\begin{frame}[allowframebreaks]
\begin{frame}
  \frametitle<presentation>{Bibliografia essenziale}
  \begin{thebibliography}
    \beamertemplatebookbibitems
  % Start with overview books.
\bibitem{Acq}
     Paolo Acquistapace.
    \newblock {\em Appunti di Analisi Funzionale}.
    \newblock http://www.dm.unipi.it/$\sim$acquistp/anafun.pdf .
\bibitem{Bes} 
    A. S. Besicovitch.
    \newblock {\em On the definitions of tangents to sets of infinite linear measure}.
    \newblock Proc. Camb. Phil. Soc. 52 (1956) 2029.
\bibitem{El-Ke}
    Marton Elekes e Tamas Keleti.
    \newblock {\em Borel sets which are null or non-$\sigma$-finite for
               every translation invariant measure}.
    \newblock Adv. Math. 201 (2006) 102-115.
\bibitem{Mo-Re2}  
    Ursula Molter e Ezequiel Rela.
    \newblock {\em Improving dimension estimates for Furstenberg-type sets}.
    \newblock Adv. Math. 223 (2010) 672-688.
  \end{thebibliography}
\end{frame}
\begin{frame}
  \frametitle<presentation>{Bibliografia essenziale}
  \begin{thebibliography}
    \beamertemplatebookbibitems
  % Start with overview books.
\bibitem{Rog}
    C. A. Rogers.
    \newblock {\em Hausdorff measures}.
    \newblock Cambridge University Press, 1970.
\bibitem{Vio}  
    Carlo Viola.
    \newblock {\em Approssimazione diofantea, frazioni continue e misure d'irrazionalità}.
    \newblock La Matematica nella Societ\`a e nella Cultura, Boll. Un. Mat. Ital. (8) 7-A, agosto 2004.
\bibitem{Wol}
    Thomas Wolff.
    \newblock {\em Recent work connected with the Kakeya problem}.
    \newblock Prospects in mathematics (Princeton, NJ, 1996), 129162, Amer. Math. Soc., Providence, RI, 1999.
  \end{thebibliography}
\end{frame}






\end{document}
