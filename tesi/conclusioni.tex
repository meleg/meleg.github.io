\chapter*{Conclusioni ed ulteriori applicazioni}
\addcontentsline{toc}{chapter}{Conclusioni ed ulteriori applicazioni}

In questo lavoro si sono mostrati i principali algoritmi per risolvere le DDE e tracciare un grafico approssimativo della soluzione.
Gli aspetti che senz'altro sarebbero da approfondire sono il caso delle DDE con il ritardo dipendente dallo stato (di cui si è dato solo un cenno), 
e il caso delle SDDE (stiff delay differential equation). Entrambi gli argomenti sono molto vasti e tutt'ora fonte di ricerca.
Per studiare questi casi è comunque necessario introdurre il concetto di passo di discretizzazione variabile, specialmente nel caso delle SDDE, 
ovvero è necessario introdurre un criterio per infittire la discretizzazione attorno ai picchi della soluzione e rilassarla quando questa 
resta con buona approssimazione costante.\\[0.5cm]
Un aspetto molto importante e sul quale si basano molte applicazioni della teoria sulle DDE è quello della stabilità dei punti di equilibrio.
Si è visto che un punto stabile per una equazione differenziale (o un sistema dinamico in generale) può diventare instabile 
a causa del ritardo (osservazione 2.1) e viceversa. Pertanto nasce la necessità di capire quando il ritardo causa in cambiamento 
qualitativo della soluzione e quindi tracciare diagrammi di biforcazione. Un'applicazione immediata è l'esempio dell'equazione logistica, 
si è infatti mostrato che il ritardo può causare l'estinzione di una specie, quindi è interessante capire da quali fattori dipende 
il ritardo (presenza di predatori, cibo, clima, ...). Sulla stessa linea si basano molte applicazioni in medicina, infatti uno degli 
articoli più famosi sulle DDE è quello di Culshaw-Ruan ``A delay-differential equation model of HIV infection of $\mbox{CD4}^+$ T-cells''  
dove viene proposto un modello matematico basato sulle DDE per descrivere il comportamento del virus HIV.\\[0.3cm]
Pertanto quello delle DDE è un campo molto vasto e pieno di applicazioni, personalmente spero di poter in futuro lavorarci ancora 
e approfondire questi aspetti.
