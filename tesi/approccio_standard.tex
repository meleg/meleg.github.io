\chapter{Approccio standard per DDE}
Per approccio standard si intende l'utilizzo dei metodi numerici continui per la risoluzione delle DDE.
Come si è visto nel capitolo precedente, data una suddivisione 
$$\Delta= \left \{ t_0, t_1, \dots , t_n , \dots , t_N = t_f \right \}$$
risolvere una DDE equivale a risolvere $N$ IVP con soluzione continua.
$$
\begin{cases}
 w_{n+1}'(t)= f \left(t, w_{n+1}(t), x(t-\tau(t,w_{n+1}(t))) \right)	\hspace{0.5cm}	t_n \le t \le t_{n+1}	\\
 w_{n+1}(t_n)=y_n
\end{cases} \hspace{1.5cm}
0 \le n \le N-1
$$
dove
$$
y_n:=y(t_n)	\hspace{2cm}
x(s):=
\begin{cases}
 \phi(s)	\hspace{2.2cm}	s \le t_0			\\
 w_{i}(s)	\hspace{1cm}	t_{i-1} \le s \le t_i 	
\end{cases}
$$
Daltronde nel momento in cui utilizziamo un metodo numerico con ordine di consistenza $p$ chiediamo che la soluzione $y$ abbia le prime 
$p$ derivate continue e questo non è vero per le DDE.\\
Per vederlo è sufficiente riprendere l'esempio 2.1, in questo caso abbiamo una discontinuità già nella derivata prima, infatti $y'(0)^-=0$ 
mentre $y'(0)^+=-y(-1)=-1$, quindi la derivata prima è discontinua in $0$. Ripetendo lo stesso ragionamento troviamo che $y''(t)=-y'(t-1)$, 
quindi $y''(1)^-=-1$ mentre $y''(1)^+=1$, in generale si trova che $y^{(k+1)}$ ha una discontinuità in $k$.

\begin{exm}[Ritardo costante]
In generale la soluzione di una DDE con ritardo costante
$$
\begin{cases}
 y'(t) = f(t,y(t),y(t- \tau))		\hspace{2cm}	t_0 \le t \le t_f \\
 y(t)=\phi(t)				\hspace{5cm}	t \le t_0
\end{cases} 
$$
può avere le derivate discontinue nei punti $t_0 + k \tau$ al variare di $k \in \mathbb{Z}$;
 in particolare $y^{(k+1)}$ può evere una discontinuità nel punto $t_0+ k \tau$.\\[0.4cm]
Pertanto per risolvere una DDE con ritardo costante certamente è necessario scegliere, ai fini della consistenza del metodo, 
una suddivisione $\Delta$ che contenga i punti $t_0 + k \tau \le t_f$. 
\end{exm}

\begin{defn}
 Il punto $\xi$ è una discontinuità di ordine $k$ se per $0 \le s \le k-1$ si ha che $y^{(s)}(\xi)$ esiste e $y^{(k)}$ è lipschitziana 
e continua in un intorno di $\xi$, mentre $y^{(k+1)}$ è discontinua in $\xi$ 
\end{defn}

Nelle DDE si verifica il fenomeno della ``propagazione delle discontinuità'' , in particolare se la soluzione ha una discontinuità 
di ordine $k$ allora ne ha anche una di ordine $k+1$ a causa del ritardo. D'ora in avanti con il termine discontinuità si intenderà 
discontinuità di grado $k$.
\vspace{0.5cm} \\
Pertanto quando si utilizza un metodo consistente di ordine $p$ è necessario includere nella suddivisione $\Delta$ le discontinuità 
fino all'ordine $p-1$.

\section{Localizzazione delle discontinuità}
Si consideri una DDE della forma
$$
\begin{cases}
 y'(t) = f(t,y(t),y(t- \tau (t,y(t)))	\hspace{1cm}	t_0 \le t \le t_f \\
 y(t)=\phi(t)				\hspace{5cm}	t \le t_0
\end{cases}
$$
allora il numero e la collocazione delle discontinuità delle soluzione dipende dal comportamento della funzione
$$
\alpha(t) = t - \tau (t,y(t))
$$
Questa funzione è chiamata argomento deviato. 
\vspace{0.5cm} \\
Si osserva subito che $\alpha(t) \le t$ essendo il ritardo una funzione non negativa. Per esaminare il caso generale, supponiamo 
$\alpha(t)>t_0$ e che la storia non si incolli alla soluzione in modo liscio, nel senso che in un intorno di $t_0$ la soluzione 
non è derivabile con continuità, ovvero $\phi'(t_0)^- \ne y'(t_0)=f(t_0,\phi(t_0),\phi(\alpha(t_0)))$. Non è difficile vedere 
che se le funzioni $f,\phi, \alpha$ sono continue allora $y'$ è continua per $t > t_0$.\\
Se inoltre queste funzioni sono anche derivabili allora esiste $y''$ per ogni valore $t$ eccetto i punti $\xi$ tali che sono radici 
semplici dell'equazione $\alpha(t)=t_0$, infatti formalmente
$$
y''(t) = \frac{\partial f}{\partial t} \left( t,y(t),y(\alpha(t)) \right) + 
 \frac{\partial f}{\partial y} \left( t,y(t),y(\alpha(t)) \right) y'(t)+
 \frac{\partial f}{\partial x} \left( t,y(t),y(\alpha(t)) \right) y'(\alpha(t)) \alpha'(t)
$$
quindi si avrà
\renewcommand\arraystretch{2.5}
$$
\begin{array}{lc}
\displaystyle
y''(\xi)^+ = \frac{\partial f}{\partial t} \left( \xi,y(\xi),y(\alpha(\xi)) \right) + 
 \frac{\partial f}{\partial y} \left( \xi,y(\xi),y(\alpha(\xi)) \right) y'(\xi)+
 \frac{\partial f}{\partial x} \left( \xi,y(\xi),y(\alpha(\xi)) \right) y'(t_0)^+ \alpha'(\xi)
\\
\displaystyle
y''(\xi)^- = \frac{\partial f}{\partial t} \left( \xi,y(\xi),y(\alpha(\xi)) \right) + 
 \frac{\partial f}{\partial y} \left( \xi,y(\xi),y(\alpha(\xi)) \right) y'(\xi)+
 \frac{\partial f}{\partial x} \left( \xi,y(\xi),y(\alpha(\xi)) \right) y'(t_0)^- \alpha'(\xi)
\end{array}
$$
\renewcommand\arraystretch{1}
Dato che $\alpha'(\xi) \ne 0$ ($\xi$ è radice semplice di $\alpha(t)=t_0$) allora $y''(\xi)^+ \ne y''(\xi)^-$ quindi $y''$ ha una discontinuità 
in $\xi$. Chiamiamo $\xi$ discontinuità primaria di primo livello, per comodità indicizziamo e denotiamo con $\xi_{1,i}$ le discontinuità 
primarie al primo livello, ovvero le radici semplici di $\alpha(t)=t_0$, ovvero
$$
\forall i
\hspace{1cm}
\alpha(\xi_{1,i}) = t_0
\hspace{1cm}
\mbox{e}
\hspace{1cm}
\alpha'(\xi_{1,i}) \ne 0
$$
come anticipato prima le discontinuità si propagano a causa del ritardo e non è difficile vedere che le discontinuità primarie 
al primo livello danno lugo ad eventuali discontinuità di $y'''$ (se esiste), è possibile ripetere lo stesso ragionamento e cercare 
le radici semplici di $\alpha(t)=\xi_{1,i}$ e verificare con gli stessi conti che $y'''$ è discontinua in tali punti. 
Chiamiamo le radici semplici di $\alpha(t)=\xi_{1,i}$ discontinuità primarie al secondo livello. \\[0.3cm]
Si definiscono discontinuità primarie al $k$-esimo 
livello le radici semplici di $\alpha(t)=\xi_{k-1,i}$ e questi sono punti in cui $y^{(k+1)}$ (se esiste) potrebbe esser discontinua.
E' chiaro che le discontinuità di ordine $p$ sono incluse nell'insieme delle discontinuità primarie al livello $p$, pertanto 
la richiesta che si farà sulla suddivisione, se il metodo è consistente di ordine $p$, sarà
$$
\xi_{i,j} \in \Delta
\hspace{1cm}
\mbox{con}
\hspace{1cm}
j \le p-1
\hspace{0.5cm}
\mbox{,}
\hspace{0.5cm}
\xi_{i,j} \le t_f
$$
Daltronde trovare $\xi_{i,j}$ può esser particolarmente difficile, si analizzerà a fondo il caso $\tau=\tau(t)$ e si darà un'idea 
di come affrontare $\tau=\tau(t,y(t))$.
\section{Ritardo dipendente dal tempo}
Si consideri una DDE della forma
$$
\begin{cases}
 y'(t) = f(t,y(t),y(t- \tau (t))	\hspace{2.5cm}	t_0 \le t \le t_f \\
 y(t)=\phi(t)				\hspace{5cm}	t \le t_0
\end{cases}
$$
In questo caso è possibile risolvere la DDE usando il metodo dei passi come indicato dal seguente algoritmo

\begin{itemize}

\item[1] \textit{Localizzazione delle discontinuità} \vspace{0.5cm}	\\
	Risolvere
	    $$
	    \begin{cases}
    	    \xi_{k,j} - \tau (\xi_{k,j}) = \xi_{k-1,i}\\
	    \xi_{0,1}=t_0
	    \end{cases}
	    \hspace{2cm}
	    1 \le i \le p	\hspace{0.5cm} , \hspace{0.5cm}	\xi_{k,j} \le t_f
	    $$
	e ordinare le discontinuità in modo da avere
	  $$
	  \xi_1	< \dots	< \xi_s
	  $$
      mentre  $\xi_0=t_0$, $\xi_{s+1}=t_f$
\item[2] \textit{Calcolo della soluzione discreta} \vspace{0.5cm}	\\
	 Risolvere con un metodo discreto con ordine di consistenza $p$
	  $$
	  \begin{cases}
	   z'(t) = f(t,z(t),\phi(t-\tau(t)))	\hspace{1cm}	\xi_0 \le t \le \xi_1	\\
	   z(\xi_0)=\phi(\xi_0)
	  \end{cases}
	  $$

\item[3] \textit{Iterare per $i=1, \dots, s$} \vspace{0.5cm}
	  \begin{itemize}
	   \item 
		Calcolare e memorizzare l'estensione continua $\eta(t)$ in $[\xi_{i-1},\xi_i]$ interpolando la soluzione solo in $ [\xi_{i-1},\xi_i]$
	  
	   \item
		Risolvere con il metodo discreto
			  $$
			  \begin{cases}
			  z'(t) = f(t,z(t),\eta(t-\tau(t)))	\hspace{1cm}	\xi_i \le t \le \xi_{i+1}	\\
			  z(\xi_i)=\phi(\xi_i)
			  \end{cases}
			  $$
	  \end{itemize}
\item[4] \textit{Tracciare il grafico di $\eta(t)$ in $[\xi_0,\xi_{s+1}]$}

\end{itemize}

La convergenza del metodo dei passi segue dal seguente teorema

\begin{thm}[Convergenza del metodo dei passi]
 Sotto le seguenti ipotesi
  \begin{itemize}
   \item La suddivisione $\Delta$ contiene le discontinuità presenti in $[t_0,t_f]$  di ordine al più $p$, detontate con $\xi_1, \dots, \xi_s$
   \item Il metodo numerico è consitente di ordine $p$ ed è 0-stabile
   \item L'interpolante ha ordine di consistenza $q$
   \item L'interpolazione avviene in $[t_{n-i_n},t_{n+j_n+1}] \subseteq [\xi_i, \xi_{i+1}]$
  \end{itemize}
Allora allora il metodo è convergente con ordine $q' = \min \left \{ p,q+1 \right \}$
$$
\max_{1 \le n \le N} \| y(t_n) - y_n \| = O(h^{q'})
$$
e anche l'interpolatore converge con lo stesso ordine
$$
\max_{t_0 \le t \le t_f} \| y(t) - \eta(t) \| = O(h^{q'})
$$
\end{thm}

Per dimostrare il teorema si useranno i seguenti lemmi 

\begin{lemma}
 Siano $\left \{ y_n^{\alpha} \right \}_{n=0}^N$ e $\left \{ y_n^{\beta} \right \}_{n=0}^N$ le soluzioni numeriche ottenute 
 con un metodo $0$-stabile dei seguenti IVP

$$
\begin{array}{cc}
 \begin{cases}
    y'(t)=g(t,y(t))	\\
    y(a)=\alpha
 \end{cases}
&
 \begin{cases}
    y'(t)=g(t,y(t))	\\
    y(a)=\beta
 \end{cases}
\end{array}
\hspace{1cm}
a \le t \le b
$$
Esistono due costanti $M$ ed $S$  indipendenti dalla suddivisione $\Delta$ tali che
\renewcommand\arraystretch{1,5}
$$
\begin{array}{lc}
\displaystyle
\max_{k \le n \le N} \| y_n^\alpha - y_n^\beta \| \le M \max_{0 \le s \le k-1} \| y_s^\alpha - y_s^\beta \|	\\
\displaystyle
\max_{a \le t \le b} \| \eta^\alpha(t) - \eta^\beta(t) \| \le S \max_{0 \le s \le k-1} \| y_s^\alpha - y_s^\beta \|
\end{array}
$$
\renewcommand\arraystretch{1}


\end{lemma}

\begin{lemma}
 Siano $\left \{ y_n \right \}_{n=0}^N$ e $\left \{ z_n \right \}_{n=0}^N$ le soluzioni numeriche ottenute 
 con un metodo $0$-stabile dei seguenti IVP

$$
\begin{array}{cc}
 \begin{cases}
    y'(t)=f(t,y(t),u(t))	\\
    y(a)=\alpha
 \end{cases}
&
 \begin{cases}
    y'(t)=f(t,y(t),v(t))	\\
    y(a)=\alpha
 \end{cases}
\end{array}
\hspace{1cm}
a \le t \le b
$$
Esistono due costanti $M$ ed $S$  indipendenti dalla suddivisione $\Delta$ ed un $h_0$ tale che se $h_i \le h_0$ allora
\renewcommand\arraystretch{1,5}
$$
\begin{array}{lc}
\displaystyle
 \max_{1 \le n \le N} \| y_n - z_n \| \le M \max_{1 \le s \le k-1} \| y_s - z_s \| + M \max_{a \le t \le b} \| u(t)-v(t) \|	\\
\displaystyle
 \max_{a \le t \le b} \| \eta(t) - \zeta(t) \| \le S \max_{1 \le s \le k-1} \| y_s - z_s \| + S \max_{a \le t \le b} \| u(t)-v(t) \|
\end{array}
$$
\renewcommand\arraystretch{1}
\end{lemma}

La dimostrazione di questi lemmi è soltanto tecnica e segue le stesse idee della dimostrazione del teorema 3.2, per 
i dettagli si veda \cite[lemmi 4.1.3 e 4.1.4]{2}

\begin{proof}[Dimostrazione del teorema 3.9]
 La tesi si dimostra per induzione sul numero di discontinuità $\xi_i$. \\
Nell'intervallo $[\xi_0, \xi_1]$ il problema diventa
	  $$
	  \begin{cases}
	   z'(t) = f(t,z(t),\phi(t-\tau(t)))	\hspace{1cm}	\xi_0 \le t \le \xi_1	\\
	   z(\xi_0)=\phi(\xi_0)
	  \end{cases}
	  $$
su questo intervallo, ponendo $g(t,y(t))=f(t,z(t),\phi(t-\tau(t)))$, la tesi è vera per il teorema 3.2 .	\\
Supponendo la tesi valida fino a $\xi_i$ allora bisognerà mostrare che è valida anche su $[\xi_i, \xi_{i+1}]$ dove il problema da risolvere è
$$
\begin{cases}
 w'(t) = f (t,w(t),\eta(t-\tau(t)))	\hspace{1cm}	\xi_i \le t \le \xi_{i+1}	\\
 w(\xi_i)=\eta(\xi_i)
\end{cases}
$$
Si osservi che non è possibile applicare il teorema 3.2 a questo problema dato che $\eta(t)$ è una funzione polinomiale a tratti quindi 
$\eta(t-\tau(t))$ potrebbe non essere abbastanza regolare, daltronde è possibile applicare il teorema 3.2 al seguente IVP
\begin{equation}
\begin{cases}
 x'(t) = f (t,x(t),y(t-\tau(t)))	\hspace{1cm}	\xi_i \le t \le \xi_{i+1}	\\
 x(\xi_i)=y(\xi_i)
\end{cases}
\end{equation}

ora il termine $y(t-\tau(t))$ è abbastanza regolare, quindi si avrà
\begin{equation}
\max_{\xi_i \le t \le \xi_{i+1}} \| y(t) -  \zeta(t) \| = O(h^{q'})
\end{equation}

Si consideri ora il problema ausiliario
$$
\begin{cases}
 z'(t) = f(t,z(t),\eta(t-\tau(t)))	\hspace{1cm}	\xi_i \le t \le \xi_{i+1}	\\
 z(\xi_i)=y(\xi_i)
\end{cases}
$$
Dal lemma 3.1 , posto $\rho(t)$ l'interpolatore per il problema ausiliario,  si ha che
$$
\max_{\xi_i \le t \le \xi_{i+1}} \| \rho(t) - \eta(t) \| \le \max_{0 \le s \le k-1} \| y_s^\alpha - y_s^\beta \|
$$
per ipotesi induttiva
$$
\max_{\xi_0 \le t \le \xi_i} \| y(t) - \eta(t) \| = O(h^{q'})
$$
quindi dal lemma 3.1 applicato usando il dato inziale segue che
$$
\max_{0 \le s \le k-1} \|  y_s^\alpha - y_s^\beta \| \le M \| y(\xi_i) - \eta(\xi_i) \| = O(h^{q'})
$$
e quindi
\begin{equation}
\max_{\xi_i \le t \le \xi_{i+1}} \| \rho(t) - \eta(t) \| = O(h^{q'}) 
\end{equation}

A questo punto riprendendo il problema originario (3.2) nell'intervallo $[\xi_i, \xi_{i+1}]$ usando il lemma 3.2 ponendo $u(t)=y(t-\tau(t))$ e 
$v(t)=\eta(t-\tau(t))$ , scegliendo un passo di discretizzazione abbastanza piccolo
\renewcommand\arraystretch{1,5}
$$
\begin{array}{c lc}
\displaystyle
\max_{\xi_i \le t \le \xi_{i+1}} \| \rho(t) - \zeta(t) \|  	& \displaystyle	\le	 S \max_{1 \le s \le k-1} \| y_s - z_s \| 
									+ S \max_{\xi_i \le t \le \xi_{i+1}} \| y(t-\tau(t)) - \eta(t- \tau(t)) \|
\\
								& \displaystyle \le	 S \max_{1 \le s \le k-1} \| y_s - z_s \|
									+ S \max_{t \le \xi_i} \| y(t) - \eta(t) \|
\end{array}
$$
\renewcommand\arraystretch{1}
Dato che la costante $S$ dipende solo dalla soluzione del problema $y(t)$ allora usando il lemma 3.2 si ha
$$
\max_{\xi_i \le t \le \xi_{i+1}} \| \rho(t) - \zeta(t) \| = O(h^{q'})
$$
quindi usando la formula (3.4) si ottiene
\begin{equation}
 \max_{\xi_i \le t \le \xi_{i+1}} \| \eta(t) - \zeta(t)	\| = O(h^{q'})
\end{equation}
Ora si conclude usando la disuguaglianza triangolare e in sequenza le formule (3.3) e (3.5)
\renewcommand\arraystretch{1,5}
$$
\begin{array}{c lc}
\displaystyle
 \max_{\xi_i \le t \le \xi_{i+1}}  \| y(t)- \eta(t) \| 		&	\displaystyle
									=   \max_{\xi_i \le t \le \xi_{i+1}} 	
										  \| y(t) - \zeta(t) -\eta(t) + \zeta(t) \| 			\\
								&	\displaystyle
									\le  \max_{\xi_i \le t \le \xi_{i+1}} 
										  \| y(t) - \zeta(t) \| +  
									      \max_{\xi_i \le t \le \xi_{i+1}}  \| \eta(t) - \zeta(t) \| 	\\
								&	\displaystyle
									= O(h^{q'})
\end{array}
$$
\renewcommand\arraystretch{1}


\end{proof}

\section{Ritardo dipendente dallo stato}
Si considerino le DDE della forma
$$
\begin{cases}
 y'(t) = f(t,y(t),y(t- \tau (t,y(t)))	\hspace{2.5cm}	t_0 \le t \le t_f \\
 y(t)=\phi(t)				\hspace{5.7cm}	t \le t_0
\end{cases}
$$
è possibile utilizzare l'approccio standard anche in questo caso, però è necessario fare una richiesta ulteriore, ovvero sarà 
necessario che l'interpolazione avvenga anche nell'intervallo corrente, quindi si 
chiederà che $j_n=0$. Sotto questa ulteriore ipotesi vale il teorema 3.9 \cite[par. 4.3]{2}.
Osserviamo subito che questo non è comunque soddisfacente e che computazionalmente il problema è decisamente più difficile, infatti 
ai fini della convergenza è necessario che le discontinuità della soluzione siano punti interni alla griglia $\Delta$ e come mostrato 
prima per trovare le (eventuali) discontinuità è necessario risolvere
$$
\begin{cases}
\xi_{i+1}-\tau(\xi_{i+1},y(\xi_{i+1}))= \xi_i		\\
\xi_0=t_0
\end{cases}
$$
Una possibile strategia per localizzare le discontinuità è quella di scegliere una griglia $\Delta$ in modo casuale (ad esempio equispaziata) 
 risolvere la DDE con l'approccio standard e successivamente risolvere
$$
\begin{cases}
\widehat{\xi}_{i+1}-\tau(\widehat{\xi}_{i+1},\eta(\xi_{i+1}))= \widehat{\xi}_i		\\
\widehat{\xi}_0=t_0
\end{cases}
$$
E' possibile dimostrare allora $\widehat{\xi_i}$ tende a $\xi_i$ con ordine di accuratezza che dipende dall'ordine dell'interpolatore e 
dall'ordine della discontinuità $\xi_i$. Quindi successivamente si risolverà nuovamente la DDE utilizzando come griglia $\Delta$ quella 
delle discontinuità appena calcolate.
\vspace{0.5cm}\\
E' possibile raffinare questo metodo per renderlo efficiente, l'idea è quella di `` catturare '' le discontinuità. Come detto all'inizio 
le discontinuità della soluzione da un certo $t$ in poi sono dovute alla proparazione (tramite il ritardo) di discontinuità presenti 
prima di $t$ (quindi tutto inizia dal possibile punto di discontinuità nell'incollamento con la storia), pertanto il primo punto di discontinuità 
è $t_0$ e gli altri sono dovuti alla sua propagazione.\vspace{0.5cm}\\
Si consideri dunque il problema locale
\begin{equation}
 \begin{cases}
 w_{n+1}'(t)	=	f(t,w_{n+1}(t),x(t-\tau(t,w_{n+1}(t))))		\hspace{1cm}	t_n \le t \le t_{n+1}		\\
 w_{n+1}(t_n) 	=	 y_n
\end{cases}
\end{equation}
dove
$$
x(s)=
\begin{cases}
\begin{array}{c lc}
 \phi(s)		&	\mbox{se}	\hspace{0.5cm}		s \le t_0			\\	
 \eta(s)		&	\mbox{se}	\hspace{0.5cm}		t_0 \le s \le t_n		\\
 w_{n+1}(s)		&	\mbox{se}	\hspace{0.5cm}		t_n \le s \le t_{n+1}
\end{array}
\end{cases}
$$
si assumi che ci sia una discontinuità $\xi$ nell'intervallo $[t_n, t_{n+1}]$ dovuta alla propagazione di una precedente 
discontinuità $\widehat{\xi_{i}}$ tale che
$$
\xi-\tau(\xi,w_{n+1}(\xi))=\widehat{\xi_i}
$$
Risolvendo la (3.6) è possibile controllare che l'interpolatore $\eta$ è tale che l'approssimazione dell'argomento deviato 
$t-\tau(t,\eta(t))$ attraversi o meno $\xi$, ovvero cerchiamo $\xi$ tali che
$$
\xi-\tau(\xi,\eta(\xi))=\widehat{\xi_i}
$$
Se tali $\xi$ esistono e sono nell'intervallo $ [t_n, t_{n+1} )$ allora si rigetta il passo corrente di discretizzazione e 
l'interpolatore e si sceglie un nuovo passo di discretizzazione $h_{n+1}^{\mbox{\tiny new}}$ (più piccolo)
 e si ripete lo stesso ragionamento su  $[t_n, t_n+h_{n+1}^{\mbox{\tiny new}}]$.
L'algoritmo va avanti fino a che non si trova un intervallo sufficientemente piccolo tale che
$$
\xi-\tau(\xi,\eta(\xi))=\widehat{\xi_i}
$$
non ha soluzioni in $[t_n, t_n+\bar{h}]$. A questo punto possono succedere due cose: 
$$
t_n + \bar{h} -\tau(t_n + \bar{h},\eta(t_n + \bar{h}))=\widehat{\xi_i}
$$
e in tal caso si è trovata la discontinuità, oppure la discontinuità si trova dopo $t_n + \bar{h}$, quindi si ripete lo stesso 
algoritmo su $[t_n + \bar{h},t_{n+1}]$.
\vspace{1cm}\\
L'algoritmo appena descritto in genere è molto efficiente nel senso che esperimenti numerici hanno confermato che includendo le 
approssimazioni delle discontinuità calcolate con questo metodo nella suddivisione $\Delta$ si hanno buoni risultati di convergenza in 
accordo al teorema 4.1 (esteso). Daltronde è chiaro che l'algoritmo potrebbe avere un costo computazionale troppo alto oppure 
potrebbe non funzionare nel caso in cui le discontinuità siano troppo vicine.